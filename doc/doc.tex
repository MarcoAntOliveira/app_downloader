\documentclass[letterpaper]{article}
\usepackage[legalpaper, left=1 cm, right=1cm, top=0.5cm, bottom=0.5cm] {geometry}
\date{} % Remove a exibição da data
\usepackage{xcolor}
\usepackage{listings}
\usepackage{graphicx}
\usepackage{hyperref} % Para criar links
\usepackage[utf8]{inputenc}
\usepackage[T1]{fontenc}

\definecolor{codegreen}{rgb}{0,0.6,0}
\definecolor{codegray}{rgb}{0.5,0.5,0.5}
\definecolor{codepurple}{rgb}{0.58,0,0.82}
\definecolor{backcolour}{rgb}{0.95,0.95,0.92}

\lstdefinestyle{mystyle}{
    language=HTML, % Use 'HTML' para a linguagem HTML
    basicstyle=\ttfamily\small,
    keywordstyle=\color{blue},
    stringstyle=\color{codepurple},
    commentstyle=\color{blue}\itshape,
    numbers=left,
    numberstyle=\tiny\color{orange},
    breaklines=true,
    showstringspaces=false,
}

\lstdefinestyle{pythonStyle}{
    language=Python,
    basicstyle=\ttfamily\small,
    keywordstyle=\color{blue},
    stringstyle=\color{orange},
    commentstyle=\color{orange}\itshape,
    numbers=left,
    numberstyle=\tiny\color{blue},
    breaklines=true,
    showstringspaces=false,
}

\lstdefinestyle{pythonStyle-on}{
    language=Python,
    basicstyle=\ttfamily\small\color{green},
    keywordstyle=\color{green},
    stringstyle=\color{green},
    commentstyle=\color{green}\itshape,
    numbers=left,
    numberstyle=\tiny\color{green},
    breaklines=true,
    showstringspaces=false,
}

\lstdefinestyle{pythonStyle-off}{
    language=Python,
    basicstyle=\ttfamily\small\color{red},
    keywordstyle=\color{red},
    stringstyle=\color{red},
    commentstyle=\color{red}\itshape,
    numbers=left,
    numberstyle=\tiny\color{red},
    breaklines=true,
    showstringspaces=false,
}


% Define o estilo para HTML
\lstdefinestyle{htmlStyle}{
    language=python,
    basicstyle=\ttfamily\small,
    keywordstyle=\color{blue},
    stringstyle=\color{orange},
    commentstyle=\color{blue}\itshape,
    numbers=left,
    numberstyle=\tiny\color{orange},
    breaklines=true,
    showstringspaces=false,
}

\lstdefinestyle{htmlStyle-on}{
    language=python,
    basicstyle=\ttfamily\small\color{green},
    keywordstyle=\color{green},
    stringstyle=\color{green},
    commentstyle=\color{green}\itshape,
    numbers=left,
    numberstyle=\tiny\color{green},
    breaklines=true,
    showstringspaces=false,
}

\lstdefinestyle{htmlStyle-off}{
    language=python,
    basicstyle=\ttfamily\small\color{red},
    keywordstyle=\color{red},
    stringstyle=\color{red},
    commentstyle=\color{red}\itshape,
    numbers=left,
    numberstyle=\tiny\color{red},
    breaklines=true,
    showstringspaces=false,
}


\lstdefinestyle{cssStyle}{
    language=HTML,
    basicstyle=\ttfamily\small,
    keywordstyle=\color{blue},
    stringstyle=\color{green},
    commentstyle=\color{blue}\itshape,
    numbers=left,
    numberstyle=\tiny\color{green},
    breaklines=true,
    showstringspaces=false,
}

\lstdefinestyle{cssStyle-off}{
    language=HTML,
    basicstyle=\ttfamily\small\color{red},
    keywordstyle=\color{red},
    stringstyle=\color{red},
    commentstyle=\color{red}\itshape,
    numbers=left,
    numberstyle=\tiny\color{red},
    breaklines=true,
    showstringspaces=false,
}

\lstdefinestyle{cssStyle-on}{
    language=HTML,
    basicstyle=\ttfamily\small\color{green},
    keywordstyle=\color{green},
    stringstyle=\color{green},
    commentstyle=\color{green}\itshape,
    numbers=left,
    numberstyle=\tiny\color{green},
    breaklines=true,
    showstringspaces=false,
}

\lstset{style=mystyle}
\title{\textbf{download_app/doc}}
\begin{document}
\maketitle

\section{Interface user}
\begin{lstlisting}[style=pythonStyle, title= Interface do usuario ] 
from tkinter import *
from tkinter import filedialog

root = Tk()
root.title("Video Downloader")
canvas = Canvas(root, width=400, height = 300)
canvas.pack()


#App label
app_label = Label(root, text="Video Downloader", fg="blue", font =("Arial", 30))
canvas.create_window(200, 20, window=app_label)

# entry to accept video url
url_label=Label(root,text="Enter video URL")
url_entry =Entry(root)
canvas.create_window(200, 80, window=url_label)
canvas.create_window(200, 100, window=url_entry)

root.mainloop()
\end{lstlisting}
\section{Button}
\begin{lstlisting}[style=pythonStyle, title=line 20 for 28 ]

#Path to download videos
path_label = Label(root, text = "Select path to download")
path_button =Button(root, text = "Select")#Path to download videos
path_label = Label(root, text = "Select path to download")
path_button =Button(root, text = "Select")
canvas.create_window(200, 150, window=path_label)
canvas.create_window(200, 170, window=path_button)

#download Button
download_button =Button(root, text='Download')
canvas.create_window(200, 250, window=download_button)
\end{lstlisting}

\section{Função de para path do computador}
\begin{lstlisting}[style=pythonStyle, title=linha 15 a 17 ] 
    def get_path():
    path =filedialog.askdirectory()
    path_label.config(text =path)
\end{lstlisting}

\section{Função para download}

\begin{lstlisting}[style=pythonStyle, title= 7 a 12] 
    def download():
        video_path =url_entry.get()
        file_path = path_label.cget("text")
        mp4 = YouTube(video_path).streams.get_highest_resolution().download()
        video_clip = VideoFileClip(mp4)
        video_clip.close()
\end{lstlisting}

\section{Movendo video para uma pasta externa }
\begin{lstlisting}[style=mystyle, title= ] 
    def download():
    video_path =url_entry.get()
    file_path = path_label.cget("text")
    print('Downloading ..')
    mp4 = YouTube(video_path).streams.get_highest_resolution().download()
    video_clip = VideoFileClip(mp4)
    video_clip.close()
    shutil.move(mp4, file_path)
    print("download complete")
\end{lstlisting}

\section{Final Code}
\lstinputlisting[title=demo.py]{demo.py}


\end{document}
\section{Movendo video para uma pasta externa }
\begin{lstlisting}[style=mystyle, title= ] 
    
\end{lstlisting}
\lstinputlisting[title=aula03.html]{aulas/aula03.html}

